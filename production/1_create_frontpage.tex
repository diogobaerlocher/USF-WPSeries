\documentclass[11pt]{article}

\usepackage{graphicx, geometry, setspace, pdfpages}
\usepackage[T1]{fontenc}
\renewcommand*\familydefault{\sfdefault}
\geometry{top = 0.75in, bottom = 0.75in, left = 0.75in, right = 0.75in}

\usepackage{xcolor}
\definecolor{usfgreen}{RGB}{0,103,71}
\definecolor{usfgold}{RGB}{207, 196, 147}

\begin{document}
\onehalfspacing

\noindent \includegraphics[width = 0.4\textwidth]{USouthFlorida-lightbg-2c-rgb-h.png} \\

\noindent{\color{usfgreen} \rule{0.5\textwidth}{10pt}}{\color{usfgold} \rule{0.5\textwidth}{10pt}}\\

\noindent {\Large Department of Economics} \\
\noindent {\Large Working Paper Number 2023-04} \\

\vspace*{0.5in}

\begin{center} 
    \huge Teacher Licensing, Teacher Supply, and Student Achievement: Nationwide Implementation of edTPA
\end{center}

\vspace*{0.5in}

\noindent \textbf{Abstract:} The recent controversial roll-out of the educative Teacher Performance Assessment (edTPA) - a performance-based exam - raises the bar of initial teacher licensure and makes teacher recruitment difficult. We leverage the quasi-experimental setting of different adoption timing by states and analyze multiple data sources containing a national sample of prospective teachers and students of new teachers in the US. With extensive controls of concurrent policies, we find that the edTPA reduced prospective teachers in undergraduate programs, less-selective and minority-concentrated universities. Contrary to the policy intention, we do not find evidence that edTPA increased student test scores.

\vspace*{0.25in}

\noindent \textbf{Authors:}

Bobby W. Chung, University of South Florida \\
\indent Jian Zou, UIUC

\vspace*{0.25in}


\noindent Available Online: October 2023 %| Updated: Date

\vfill 
\noindent{\color{usfgreen} \rule{\textwidth}{5pt}}\\
\noindent \small ©The authors listed. All rights reserved. No part of this paper may be reproduced in any form, or stored in a retrieval system, without the prior written permission of the authors.
\thispagestyle{empty}

\end{document}