\documentclass[11pt]{article}

\usepackage{graphicx, geometry, setspace, pdfpages}
\usepackage[T1]{fontenc}
\renewcommand*\familydefault{\sfdefault}
\geometry{top = 0.75in, bottom = 0.75in, left = 0.75in, right = 0.75in}

\usepackage{xcolor}
\definecolor{usfgreen}{RGB}{0,103,71}
\definecolor{usfgold}{RGB}{207, 196, 147}

\begin{document}
\onehalfspacing

\noindent \includegraphics[width = 0.4\textwidth]{USouthFlorida-lightbg-2c-rgb-h.png} \\

\noindent{\color{usfgreen} \rule{0.5\textwidth}{10pt}}{\color{usfgold} \rule{0.5\textwidth}{10pt}}\\

\noindent {\Large Department of Economics} \\
\noindent {\Large Working Paper Number 2025-04} \\

\vspace*{0.5in}

\begin{center} 
    \huge Racial Self-Classification, Group Consciousness, and Public Employment Representation
\end{center}

\vspace*{0.5in}

\noindent \textbf{Abstract:} This paper examines how racial identity misrepresentation influences public sector hiring in Brazil. We focus on misaligned white candidates — those who self-identify as white but are unlikely to be classified as such by facial recognition — and exploit close electoral races using a regression discontinuity design. Narrow victories by these candidates reduce the share of nonwhite hires in municipal legislative offices by approximately 20\%, with effects concentrated in temporary and managerial positions. We also find a significant decline in nonwhite leadership in municipal secretariats. These results indicate that misaligned whiteness shapes racial representation through political and bureaucratic channels.

\vspace*{0.25in}

\noindent \textbf{Authors:}

Diogo Baerlocher, University of South Florida \\
\indent Rodrigo Schneider, Skidmore College

\vspace*{0.25in}


\noindent Available Online: September 2025 %| Updated: Date

\vfill 
\noindent{\color{usfgreen} \rule{\textwidth}{5pt}}\\
\noindent \small ©The authors listed. All rights reserved. No part of this paper may be reproduced in any form, or stored in a retrieval system, without the prior written permission of the authors.
\thispagestyle{empty}

\end{document}