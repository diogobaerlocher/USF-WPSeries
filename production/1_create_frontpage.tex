\documentclass[11pt]{article}

\usepackage{graphicx, geometry, setspace, pdfpages}
\usepackage[T1]{fontenc}
\renewcommand*\familydefault{\sfdefault}
\geometry{top = 0.75in, bottom = 0.75in, left = 0.75in, right = 0.75in}

\usepackage{xcolor}
\definecolor{usfgreen}{RGB}{0,103,71}
\definecolor{usfgold}{RGB}{207, 196, 147}

\begin{document}
\onehalfspacing

\noindent \includegraphics[width = 0.4\textwidth]{USouthFlorida-lightbg-2c-rgb-h.png} \\

\noindent{\color{usfgreen} \rule{0.5\textwidth}{10pt}}{\color{usfgold} \rule{0.5\textwidth}{10pt}}\\

\noindent {\Large Department of Economics} \\
\noindent {\Large Working Paper Number 2026-01} \\

\vspace*{0.5in}

\begin{center} 
    \huge Beyond Appearance: The Socioeconomic and Historical Roots of Racial Identity in Brazil
\end{center}

\vspace*{0.25in}

\noindent \textbf{Abstract:} Racial identity is not solely a matter of physical appearance but is also shaped by social and historical context. Using data on over 500,000 candidates for local office in Brazil’s 2020 elections, we study how self-reported race---specifically, identification as \textit{white}---relates to phenotypic appearance, socioeconomic characteristics, and local social perceptions. We use machine learning to extract appearance-based probabilities of racial classification from candidate photographs and show that these probabilities explain a significant share of variation in self-reported race. Socioeconomic factors such as education, gender, and wealth also influence racial identification, though their effects diminish among individuals whose appearance more clearly aligns with the \textit{white} category. Municipality fixed effects, which we interpret as capturing local social perception bias, vary systematically across regions and are strongly associated with historical slave population shares. We further show that areas with state-sponsored European settlements---often associated with more inclusive institutions---exhibit lower rates of \textit{white} self-identification, contrasting with the positive association between slavery intensity and \textit{white} identification. Our findings highlight the enduring role of social and historical forces in shaping racial classification and suggest that racial inequality cannot be fully understood without accounting for the social construction of race.

\vspace*{0.25in}

\noindent \textbf{Authors:}

Diogo Baerlocher, University of South Florida \\
\indent Renata Caldas, University of South Florida \\
\indent Francisco Cavalcanti, Universidade Federal De Pernambuco \\

\vspace*{0.25in}


\noindent Available Online: January 2026 %| Updated: Date

\vfill 
\noindent{\color{usfgreen} \rule{\textwidth}{5pt}}\\
\noindent \small ©The authors listed. All rights reserved. No part of this paper may be reproduced in any form, or stored in a retrieval system, without the prior written permission of the authors.
\thispagestyle{empty}

\end{document}