\documentclass[11pt]{article}

\usepackage{graphicx, geometry, setspace, pdfpages}
\usepackage[T1]{fontenc}
\renewcommand*\familydefault{\sfdefault}
\geometry{top = 0.75in, bottom = 0.75in, left = 0.75in, right = 0.75in}

\usepackage{xcolor}
\definecolor{usfgreen}{RGB}{0,103,71}
\definecolor{usfgold}{RGB}{207, 196, 147}

\begin{document}
\onehalfspacing

\noindent \includegraphics[width = 0.4\textwidth]{USouthFlorida-lightbg-2c-rgb-h.png} \\

\noindent{\color{usfgreen} \rule{0.5\textwidth}{10pt}}{\color{usfgold} \rule{0.5\textwidth}{10pt}}\\

\noindent {\Large Department of Economics} \\
\noindent {\Large Working Paper Number 2024-04} \\

\vspace*{0.5in}

\begin{center} 
    \huge Boundaries Generate Discontinuities in the Urban Landscape
\end{center}

\vspace*{0.5in}

\noindent \textbf{Abstract:} Neighborhood boundaries are often determined by physical topography, transportation networks, or the administration of public goods (e.g., school attendance zones). We present a simple model of boundaries that predicts discontinuities in household demographics, the supply of amenities, and home prices at physical and administrative boundaries. We take these predictions to the data and find abundant evidence of discontinuities in a wide range of observable dimensions – the universe of variables available in the 2020 Census at the Block group level – and six different types of boundaries. We draw two important conclusions from these findings: (1) researchers should implement boundary discontinuity designs with caution because the key identification assumption may not hold except in narrow applications, and (2) even narrowly targeted place-based policies may have much broader impacts if they involve a new administrative boundary.

\vspace*{0.25in}

\noindent \textbf{Authors:}

Vikram Maheshri, University of Houston \\
\indent Kenneth Whaley, University of South Florida \\

\vspace*{0.25in}


\noindent Available Online: October 2024 %| Updated: Date

\vfill 
\noindent{\color{usfgreen} \rule{\textwidth}{5pt}}\\
\noindent \small ©The authors listed. All rights reserved. No part of this paper may be reproduced in any form, or stored in a retrieval system, without the prior written permission of the authors.
\thispagestyle{empty}

\end{document}