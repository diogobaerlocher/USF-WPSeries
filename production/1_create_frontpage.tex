\documentclass[11pt]{article}

\usepackage{graphicx, geometry, setspace, pdfpages}
\usepackage[T1]{fontenc}
\renewcommand*\familydefault{\sfdefault}
\geometry{top = 0.75in, bottom = 0.75in, left = 0.75in, right = 0.75in}

\usepackage{xcolor}
\definecolor{usfgreen}{RGB}{0,103,71}
\definecolor{usfgold}{RGB}{207, 196, 147}

\begin{document}
\onehalfspacing

\noindent \includegraphics[width = 0.4\textwidth]{USouthFlorida-lightbg-2c-rgb-h.png} \\

\noindent{\color{usfgreen} \rule{0.5\textwidth}{10pt}}{\color{usfgold} \rule{0.5\textwidth}{10pt}}\\

\noindent {\Large Department of Economics} \\
\noindent {\Large Working Paper Number 2025-01} \\

\vspace*{0.5in}

\begin{center} 
    \huge Old But Gold: Historical Pathways and Path Dependence
\end{center}

\vspace*{0.5in}

\noindent \textbf{Abstract:} Following the discovery of gold in 1694 in Brazil, pathways were constructed to connect coastal settlements to mining regions in the unpopulated interior. While these pathways initially facilitated the creation of road towns, their influence faded by the late nineteenth century. With the mid-twentieth-century demographic and industrial transition, regions with higher historical road density experienced renewed population growth and greater migrant inflows. We argue that this resurgence reflects the role of road towns in fostering early urbanization and structural transformation. Using an extended Rosen-Roback-Glaeser framework, we estimate strong agglomeration spillovers, suggesting that Brazil’s spatial economy exhibits multiple steady states and historical path dependence.
\vspace*{0.25in}

\noindent \textbf{Authors:}

Diogo Baerlocher, University of South Florida \\
\indent Diego Firmino, Universidade Federal Rural de Pernambuco \\
\indent Guilherme Lambais, Lusíada University of Lisbon \\
\indent Eust\'aquio Reis, IPEA \\
\indent Henrique Veras, Universidade Federal de Pernambuco \\


\vspace*{0.25in}


\noindent Available Online: January 2025 %| Updated: Date

\vfill 
\noindent{\color{usfgreen} \rule{\textwidth}{5pt}}\\
\noindent \small ©The authors listed. All rights reserved. No part of this paper may be reproduced in any form, or stored in a retrieval system, without the prior written permission of the authors.
\thispagestyle{empty}

\end{document}