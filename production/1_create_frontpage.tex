\documentclass[11pt]{article}

\usepackage{graphicx, geometry, setspace, pdfpages}
\usepackage[T1]{fontenc}
\renewcommand*\familydefault{\sfdefault}
\geometry{top = 0.75in, bottom = 0.75in, left = 0.75in, right = 0.75in}

\usepackage{xcolor}
\definecolor{usfgreen}{RGB}{0,103,71}
\definecolor{usfgold}{RGB}{207, 196, 147}

\begin{document}
\onehalfspacing

\noindent \includegraphics[width = 0.4\textwidth]{USouthFlorida-lightbg-2c-rgb-h.png} \\

\noindent{\color{usfgreen} \rule{0.5\textwidth}{10pt}}{\color{usfgold} \rule{0.5\textwidth}{10pt}}\\

\noindent {\Large Department of Economics} \\
\noindent {\Large Working Paper Number 2024-05} \\

\vspace*{0.5in}

\begin{center} 
    \huge Finding Home When Disaster Strikes: Dust Bowl Migration and Housing in Los Angeles
\end{center}

\vspace*{0.5in}

\noindent \textbf{Abstract:} When natural disasters strike, the impact on housing markets can be far-reaching. This paper explores the unique dynamics of natural disaster-induced migration on the housing market, focusing on the 1930s Dust Bowl migration to Los Angeles---the top migrant destination. We use U.S. Census-linked and geocoded address data to document that the arrival of Dust Bowl migrants significantly impacted the city's housing market. We show that houses inhabited by Dust Bowl migrants had lower price growth over the decade. Critically, we uncover valuation spillovers within highly granular neighborhoods, where houses inhabited by non-migrants experienced lower price growth modulated by how close they were to Dust Bowl migrants. Our analysis of potential mechanisms suggests that these effects were primarily driven by the economic vulnerability of migrants rather than generalized discrimination. Our research contributes to understanding how natural disaster-induced migration shapes housing markets and the dimensions in which climate refugees differ from other migrants.

\vspace*{0.25in}

\noindent \textbf{Authors:}

Diogo Baerlocher, University of South Florida \\
\indent Gustavo S. Cortes, University of Florida \\
\indent Vinicios P. Sant'Anna, MIT \\

\vspace*{0.25in}


\noindent Available Online: November 2024 %| Updated: Date

\vfill 
\noindent{\color{usfgreen} \rule{\textwidth}{5pt}}\\
\noindent \small ©The authors listed. All rights reserved. No part of this paper may be reproduced in any form, or stored in a retrieval system, without the prior written permission of the authors.
\thispagestyle{empty}

\end{document}